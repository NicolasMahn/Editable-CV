\documentclass[10pt,a4paper,ragged2e,withhyper]{altacv}
\geometry{left=1.25cm,right=1.25cm,top=1.5cm,bottom=1.5cm,columnsep=1.2cm}
\usepackage{expl3}
\usepackage{paracol}
\usepackage[utf8]{inputenc}
\usepackage{xstring}
\usepackage{xkeyval}

\definecolor{ColorAccent}{HTML}{5B1F36}
\definecolor{ColorAccent2}{HTML}{812C4D}
\definecolor{SlateGrey}{HTML}{2E2E2E}
\definecolor{LightGrey}{HTML}{555555}

\colorlet{heading}{ColorAccent}
\colorlet{accent}{ColorAccent2}
\colorlet{body}{LightGrey}
\colorlet{emphasis}{SlateGrey}
\colorlet{tagline}{ColorAccent}


\renewcommand{\cvItemMarker}{{\small\textbullet}}
\renewcommand{\cvRatingMarker}{\faCircle}

\photoR{3.5cm}{BewerbungsFoto2024}

% === Define global option handler ===
\newcommand{\setCVOptions}[1]{%
  \renewcommand{\CVoptions}{#1}%
  \setkeys{CV}{#1}%
}

% === Define keys === 
\makeatletter 
\define@key{CV}{language}{\def\cvLanguage{#1}} 
\define@key{CV}{subtitle}{\def\cvSubtitle{#1}} 

\define@key{CV}{skillNames}{\def\cvSkillNames{#1}} 
\define@key{CV}{skillValues}{\def\cvSkillValues{#1}} 

\define@key{CV}{programming}{\def\cvProgramming{#1}} 
\define@key{CV}{management}{\def\cvManagement{#1}} 
\define@key{CV}{tools}{\def\cvTools{#1}} 
\define@key{CV}{masterSpecialization}{\def\cvMasterSpecialization{#1}} 
\define@key{CV}{bachelorSpecialization}{\def\cvBachelorSpecialization{#1}} 

\makeatother

%===================================================================================================================================
%-----------------------------------------------CHOOSE CONFIG FILE HERE-------------------------------------------------------------
%===================================================================================================================================
\setCVOptions{
    language={english},
    subtitle={ML Systems \& Scalable AI},
    skillNames={Machine Learning,Software Development,Applied Data Science,Software Architecture,MLOps},
    skillValues={3.5, 4, 4.5, 3, 3},
    programming={Python, PyTorch, Prompt Engineering, Wandb, TensorFlow, REST, JAX, SQL, MongoDB, Scikit-learn, R, Java},
    management={Agile Mindset, MLOps, CI/CD},
    tools={Cursor, PyCharm, VS Code, \\, Jupyter Notebook, AWS, Docker, \\, Atlassian Suite, LaTeX, Git, GitHub},
    masterSpecialization={Data Science, Machine Learning, Data Visualization, DQN, DevOps, Cloud Systems, Arch Patterns, Data Mining},
    bachelorSpecialization={Software Development, Data Science, RL}
}
%===================================================================================================================================
%-----------------------------------------------CHOOSE CONFIG FILE HERE-------------------------------------------------------------
%===================================================================================================================================

\ExplSyntaxOn

\tl_set:Nx \l_cv_skillnames_tl { \cvSkillNames } 
\tl_set:Nx \l_cv_skillvalues_tl { \cvSkillValues } 
\tl_set:Nx \l_cv_programming_tl { \cvProgramming } 
\tl_set:Nx \l_cv_management_tl { \cvManagement } 
\tl_set:Nx \l_cv_tools_tl { \cvTools } 
\tl_set:Nx \l_cv_masterSpecialization_tl { \cvMasterSpecialization } 
\tl_set:Nx \l_cv_bachelorSpecialization_tl { \cvBachelorSpecialization } 

\seq_set_split:NnV \l_cv_skillnames_seq { , } \l_cv_skillnames_tl
\seq_set_split:NnV \l_cv_skillvalues_seq { , } \l_cv_skillvalues_tl
\seq_set_split:NnV \l_cv_programming_seq { , } \l_cv_programming_tl
\seq_set_split:NnV \l_cv_management_seq { , } \l_cv_management_tl
\seq_set_split:NnV \l_cv_tools_seq { , } \l_cv_tools_tl
\seq_set_split:NnV \l_cv_masterSpecialization_seq { , } \l_cv_masterSpecialization_tl
\seq_set_split:NnV \l_cv_bachelorSpecialization_seq { , } \l_cv_bachelorSpecialization_tl


% Define a rendering command
\NewDocumentCommand{\cvtags}{m}{
  \seq_if_empty:cTF { l_cv_#1_seq }
    { } % do nothing if empty
    {
      \seq_map_inline:cn { l_cv_#1_seq } {
        \str_if_eq:nnTF {##1}{\\}
          { \\ } % line break if the item is literally "\\"
          { \cvtag{\tl_trim_spaces:n {##1}} } % otherwise render tag
      }
    }
}


\NewDocumentCommand{\cvtagstitle}{mm}{
  \seq_if_empty:cTF { l_cv_#1_seq }
    { } % do nothing if empty
    {
      \par\medskip
      \divider
      {\cvsubsection{\textcolor{ColorAccent2}{#2}}}\par\vspace{0.3em}
      \seq_map_inline:cn { l_cv_#1_seq } {
        \str_if_eq:nnTF {##1}{\\}
          { \\ } % if item is literally "\\", make a line break
          { \cvtag{\tl_trim_spaces:n {##1}} } % otherwise render tag
      }
    }
}

\NewDocumentCommand{\cvskills}{mm}{
  % #1 = skill names key, #2 = skill values key
  \seq_if_exist:cTF { l_cv_#1_seq }{
    \seq_if_exist:cTF { l_cv_#2_seq }{
      % determine how many pairs we can print (min length)
      \int_zero_new:N \l_tmpa_int
      \int_zero_new:N \l_tmpb_int
      \int_zero_new:N \l_tmpc_int
      \int_set:Nn \l_tmpa_int { \seq_count:c { l_cv_#1_seq } }
      \int_set:Nn \l_tmpb_int { \seq_count:c { l_cv_#2_seq } }
      \int_set:Nn \l_tmpc_int { \int_min:nn { \l_tmpa_int } { \l_tmpb_int } }

      % output interleaved pairs
      \int_step_inline:nn { \l_tmpc_int } {
        \cvskill{\seq_item:cn { l_cv_#1_seq } { ##1 }}{\seq_item:cn { l_cv_#2_seq } { ##1 }}
      }
    }{
      % if second list doesn't exist
      \textcolor{red}{Error: list '#2' not found.}
    }
  }{
    % if first list doesn't exist
    \textcolor{red}{Error: list '#1' not found.}
  }
}


\ExplSyntaxOff


\begin{document}

\newpage
\IfStrEq{\cvLanguage}{german}{
  \name{Nicolas Mahn}
\tagline{Angehender M.Sc. – \cvSubtitle}
\personalinfo{%
  \email{Nicolas.Mahn@gmx.de}
  \phone{+49 152 065 01315}
  \linkedin{nicolas-mahn-5453691b7}   
  \location{Stuttgart}
  \birthdate{07.08.2000}
  \github{NicolasMahn}
}

\makecvheader

\columnratio{0.6}
\begin{paracol}{2}

\cvsection{Bildung}

\cveventloccert{Doppel-M.Sc. in Business \\ Application Architecture \& Software Technology}{Hochschule Furtwangen \& Linnaeus University}{03/2023 -- heute}{Deutschland \& Växjö, Schweden}{\textit{\textbf{Vorläufige Note:} 1,6}}{https://1drv.ms/b/c/8824be59db844366/EVao29u-Ms9BrkDrdK_QAP4BdihWRbbwBJC6RHm2se4Bww?e=kKBBYj}{Notenspiegel}
\cvtag{LLM-Agents}
\cvtag{RAG}
\cvtag{Prompt Engineering}
\cvtag{Data Science}
\cvtag{ML}
\cvtag{Datenvisualisierung}
\cvtag{DQN}
\cvtag{DevOps}

\divider

\cveventloccert{B.Sc. in Wirtschaftsinformatik}{Hochschule Furtwangen}{09/2019 -- 01/2023}{Furtwangen, Deutschland}{\textbf{Note:} 1,4}{https://1drv.ms/b/c/8824be59db844366/EWZDhNtZviQggIjapQEAAAABPe8w2fviR-vc7xof-iW7Qw?e=SCFDqm}{Bachelorzeugnis}
\cvtag{Softwareentwicklung}
\cvtag{Data Science}
\cvtag{RL}

\vspace{10px}
\cvsection{Berufserfahrung}

\cveventcert{Data Scientist | Wissenschaftlicher Mitarbeiter}{Hochschule Furtwangen (KISS)}{04/2024 -- 08/2024}{https://1drv.ms/b/c/8824be59db844366/ERafVY1nI8JHoJQP-z1TVJUBrIODdwkKStv8KNcFK9IBRQ?e=jCKMLW}{Werkstudentenzeugnis}
\vspace{-7pt}
\divider
\cveventcert{Softwareentwickler | Wissenschaftlicher Mitarbeiter}{Hochschule Furtwangen (MIR)}{04/2023 -- 04/2024}{https://1drv.ms/b/c/8824be59db844366/ERafVY1nI8JHoJQP-z1TVJUBrIODdwkKStv8KNcFK9IBRQ?e=jCKMLW}{Werkstudentenzeugnis}
\vspace{-7pt}
\divider
\cveventcert{Praktikum: Digital Finance}{KPMG}{01/2022 -- 09/2022}
{https://1drv.ms/b/c/8824be59db844366/EWZDhNtZviQggIg2owEAAAABJCphwILRMJ2lanay7MEuAQ?e=CeMCE0}{Praktikumszeugnis}
\vspace{-7pt}
\divider
\cveventcert{Werkstudent}{DB Systel}{11/2021 -- 03/2022}{https://1drv.ms/i/c/8824be59db844366/EWZDhNtZviQggIjTmwEAAAABX5fHoF85jFbZlirmSiz5YA?e=Dsx0bu}{Werkstudentenzeugnis}
\vspace{-7pt}
\divider
\cveventcert{Praktikum: Digitale Schiene}{DB Netz AG}{04/2021 -- 09/2021}{https://1drv.ms/b/c/8824be59db844366/EWZDhNtZviQggIg1owEAAAABG4BePHm4Whz_tvitgg1E3w?e=U6KSz3}{Praktikumszeugnis}

\vspace{10px}
\cvsection{Publikationen}

\cveventurl{RAG VotingAid \textit{(angenommen)}}{Linnaeus University}{2025}{https://votingaid.nicolas-mahn.de/}{VotingAid}

\divider

\cveventurl{KI-gestützte mehrstufige Radweg-Analyse: Stufe 1 – Oberflächenerkennung}{Hochschule Furtwangen (MIR)}{2023}{https://link.springer.com/chapter/10.1007/978-3-031-42532-5_5}{Springer}

\switchcolumn

\cvsection{Fähigkeiten}

\cvskills{skillnames}{skillvalues}

\cvtagstitle{programming}{Programmieren}
\cvtagstitle{management}{Projektmanagement Philosophie}
\cvtagstitle{tools}{Tools}

\vspace{10px}
\cvsection{Sprachen}
\cvskill{Deutsch}{5}
\cvskill{Englisch}{4.5}
\cvskill{Französisch}{3}

\vspace{10px}
\cvsection{Ausland}

\cvevent{Erasmus Student}{Linnaeus University}{09/2024 -- 02/2025}{Växjö, Schweden}
\vspace{-7pt}
\divider

\cvevent{weltwärts Freiwilligendienst}{Kawempe Youth Development \\ Association \& Makerere West Valley Primary School}{
08/2018 -- 08/2019}{Kampala, Uganda}
\vspace{-7pt}
\divider
\cvevent{Austauschschüler}{Collège Charles de Foucault}{03/2013 -- 08/2013}{Lyon, Frankreich}
\vspace{-7pt}
\divider
\cvevent{Austauschschüler}{Sion Mills Primary School}{03/2009 -- 08/2009}{Sion Mills, Nordirland}

\end{paracol}

}{
  
\name{Nicolas Mahn}
\tagline{M.Sc. Candidate – \cvSubtitle}
\personalinfo{%
  \email{Nicolas.Mahn@gmx.de}
  \phone{+49 152 065 01315}
  \linkedin{nicolas-mahn-5453691b7}   
  \location{Stuttgart, Germany}
  \birthdate{07.08.2000}
  \github{NicolasMahn}
}

\makecvheader

\columnratio{0.6}
\begin{paracol}{2}

\cvsection{Education}

\cveventloccert{Double M.Sc. in Business \\ Application Architecture \& Software Technology}{Furtwangen University \& Linnaeus University}{03/2023 -- present}{Germany \& Växjö, Sweden}{\textit{\textbf{Preliminary Grade:} 1.6}}{https://1drv.ms/b/c/8824be59db844366/EVao29u-Ms9BrkDrdK_QAP4BdihWRbbwBJC6RHm2se4Bww?e=kKBBYj}{Report Card}
\cvtags{masterSpecialization}
%\textbf{Focus Areas:} LLM-Agents, RAG, Data Science, ML \& Prompt Engineering, LLMs, NLP, Data Visualization, DQN, and DevOps

\divider

\cveventloccert{B.Sc. in Business Informatics}{Furtwangen University}{09/2019 -- 01/2023}{Furtwangen, Germany}{\textbf{Grade:} 1.4}{https://1drv.ms/b/c/8824be59db844366/EWZDhNtZviQggIjapQEAAAABPe8w2fviR-vc7xof-iW7Qw?e=SCFDqm}{Bachelors Certificate}
\cvtags{bachelorSpecialization}
%\textbf{Focus Areas:} Software Development, Data Science, and RL


\vspace{10px}
\cvsection{Work Experience}

\cveventcert{Data Scientist | Research Associate}{Furtwangen University (KISS)}{04/2024 -- 08/2024}
{https://1drv.ms/b/c/8824be59db844366/ERafVY1nI8JHoJQP-z1TVJUBrIODdwkKStv8KNcFK9IBRQ?e=jCKMLW}{Reference Letter}
\vspace{-7pt}
\divider
\cveventcert{Software Developer | Research Associate}{Furtwangen University (MIR)}{04/2023 -- 04/2024}
{https://1drv.ms/b/c/8824be59db844366/ERafVY1nI8JHoJQP-z1TVJUBrIODdwkKStv8KNcFK9IBRQ?e=jCKMLW}{Reference Letter}
\vspace{-7pt}
\divider
\cveventcert{Internship: Digital Finance}{KPMG}{01/2022 -- 09/2022}
{https://1drv.ms/b/c/8824be59db844366/EWZDhNtZviQggIg2owEAAAABJCphwILRMJ2lanay7MEuAQ?e=CeMCE0}{Reference Letter}
\vspace{-7pt}
\divider
\cveventcert{Working Student}{DB Systel}{11/2021 -- 03/2022}{https://1drv.ms/i/c/8824be59db844366/EWZDhNtZviQggIjTmwEAAAABX5fHoF85jFbZlirmSiz5YA?e=Dsx0bu}{Reference Letter}
\vspace{-7pt}
\divider
\cveventcert{Internship: Digital Rail Program}{DB Netz AG}{04/2021 -- 09/2021}{https://1drv.ms/b/c/8824be59db844366/EWZDhNtZviQggIg1owEAAAABG4BePHm4Whz_tvitgg1E3w?e=U6KSz3}{Reference Letter}


\vspace{10px}
\cvsection{Publications}

\cveventurl{RAG VotingAid \textit{(accepted)}}{Linnaeus University}{2025}{https://votingaid.nicolas-mahn.de/}{VoitingAid}

\divider

\cveventurl{AI-Powered Multi-Stage Bike Path Analysis: Stage 1 – Surface Detection}{Hochschule Furtwangen (MIR)}{2023}{https://link.springer.com/chapter/10.1007/978-3-031-42532-5_5}{Springer}

\switchcolumn

\cvsection{Skills}

\cvskills{skillnames}{skillvalues}

\cvtagstitle{programming}{Programming}
\cvtagstitle{management}{Project Management Philosophy}
\cvtagstitle{tools}{Tools}

\vspace{10px}
\cvsection{Languages}
\cvskill{German}{5}
\cvskill{English}{4.5}
\cvskill{French}{3}

  
\vspace{10px}
\cvsection{Stays Abroad}

\cvevent{Erasmus Student}{Linnaeus University}{09/2024 -- 02/2025}{Växjö, Sweden}
\vspace{-7pt}
\divider

\cvevent{Volunteer Teacher}{Kawempe Youth Development Association \& Makerere West Valley Primary School}{
08/2018 -- 08/2019}{Kampala, Uganda}
\vspace{-7pt}
\divider
\cvevent{Exchange Student}{Collège Charles de Foucault}{03/2013 -- 08/2013}{Lyon, France}
\vspace{-7pt}
\divider
\cvevent{Exchange Student}{Sion Mills Primary School}{03/2009 -- 08/2009}{Sion Mills, N. Ireland}



\end{paracol}


}


\end{document}